\chapter{Anhang}
\section{Anhang A}
\subsection{Afgabebeschreibung}
Ausgangssituation 2 (für zwei oder drei Bearbeiterinnen)
\\
Si sind Abteilungsleiter(in) bei der Wolfsburger Ersatzkasse (WEK), einer mitgliederstarken Krankenkasse.
Ihre Abteilung  ist für den Bereich der Organisation und Qualitätssicherung  von Reha-Kurzen zuständig.Sie schließen die Verträge  mit den Reha-Kliniken, überwachen die Verlagseinhaltung, die Qualität  der Behandlung und den Behandlungserfolg. Sie betreuen neben den zahlreichen Vertragspartner auch zwei im Besitz der WEK befindliche Kurkliniken.
Aufgrund des verstärkten Wettbewerbs mit anderen Krankenkassen um der Versicherte, hat der Vorstand der WEK eine Qualitätsoffensive im Bereich der Rehakuren beschlossen. Auf der letzten Vorstandssitzung haben Sie dazu einen umfassenden Bericht  vorgelegt und Vorschläge  zu Umstrukturierung  gemacht. Im Mittelpunkt ihres Berichts stand der Bereich der orthopädischen  Reha-Maßnahmen.
Ihrer Analyse zufolge ist die WEK  in diesem Bereich durch die Politik des alten Vorstands deutlich gegenüber den Mitbewerbern im Nachteil. Die WEK kann, anders als viele andere Kassen, ihren Mitgliedern weder in medizinischer noch im Hinblick das Kurumfeld (Ausstattung der Zimmer, Unterhaltungsangebot in der Kurorten) eine ansprechende Leistung bieten. Zwar zahlen sie zurzeit sehr geringe Preise für die Kuren, der Behandlungserfolg der meisten Kurklinken ist aber zweifelhaft, so dass hohe Kosten im Bereich der Nachsorge entstehen. Ihren Berechnungen nach entstehen  so für die WEK insgesamt höhere Behandlungskosten als bei vergleichbaren Krankenkassen. Als wesentliche Ursache dafür sehen Sie den Beschluss des alten Vorstandes, bei möglichst vielen verschiedenen Klinikbetreibern möglichst kleine Kontingente zu buchen und dabei vor allem auf den Faktor „Kosten“ zu setzen. So ist der Einfluss der WEK auf die tatsächlich Behandlungsstandards in den Kliniken zum Nachteil der Kasse und auch der Versicherten verloren gegangen.
Sie haben dem Vorstand eine grundsätzliche Neuorientierung vorgeschlagen: Die Zusammenarbeit soll auf wenige Kurklinken konzentriert werden, so dass die WEK ihren Einfluss auf moderne und erfolgreiche Behandlungsstandards geltend machen kann. Dafür sollen  auch höhere Kosten für einzelne Behandlung in Kauf genommen werden. Dies rechnen sich durch eine Verringerung der Gesamtkosten durch Wegfall kostspieliger Nachbehandlungen. Zudem soll mit dem neuen Rehakonzept offensiv in das Marketing eingestiegen werden, um neue, auf Qualität bedachte Mitglieder  zu gewinnen.
Der Vorstand ist ihrer Analyse und Ihren Vorschlägen in allen Punkten gefolgt und hat Sie mit der Neuausrichtung des Bereichs „orthopädische Reha-Maßnahmen“ beauftragt.
\\
in der nächsten Woche haben Sie einen Termin mit Dr. Kneip, der zugleich Bürgermeister des Kurortes Bad Fünfquelle und neuer Geschäftsführer der Bad Fünfquell Kurbetriebs GmbH, einer 100%-Tochter der Stadt,ist.
Dieser traditionsreiche Kurort hat in den letzten Jahren nicht mit der Entwicklung Schritt gehalten. Konsequenz für Bad Fünfquell sind deutlich rückläufige Gästezahlen und damit sinkende einnehmen sowohl für die  Kurbetriebs GmbH wie auch für die Gemeinde. Die Kurbetriebs GmbH betreiben insgesamt fünf Kurkliniken mit ca. 1000 Betten, wobei jede der Klinken auf einen eigenen Reha-Bereich spezialisiert ist. Nur eine, sehr kleine Klink behandelt orthopädische Patienten. Alle Kliniken sind stark modernierungsbedürftig.
Um die Finanzierung der Modernisierung sicherzustellen, hat Dr. Kneip mit Ihren Kontakt aufgenommen. Er strebt einen umfangreichen Vertrag mit der Wolfsburger Ersatzkasse an , die ihm eine Belegung der Kurplätze sichert. Nur wenn dies gesichert ist, wird die Bank die notwendigen Kredite für die Kurbetriebs GmbH freigeben.
Aus der Presse wissen Sie, das die Pläne von Dr. Kneip in Bad Fünfquell nicht unumstritten sind. Sowohl im Standart wie auch in der Öffentlichkeit wird bezweifelt, dass die Abhängigkeit von einer einzigen großen Krankenkasse eine gute Strategie ist. Befürchtet wird vor allem , dass die Krankenkasse langfristig ihre Position ausnutzen und die Preise für die Kurbehandlung drücken wird, so dass sich die Investitionen der Gemeinde überhaupt nicht rechnen. Die Opposition im Gemeinderat seht angesichts der ständigen Androhungen in der Gesetzgebung dem Plan von Dr. Kneip skeptisch gegenüber. Sie schlägt vor, stärker auf den Bereich „Wellness“ für selbstzahlende Kundinnen und auf Kuren für Privatpatientinnen zu setzen. 
