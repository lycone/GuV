\chapter{Alternativen in Verhlandlungen}
Die festgestellte Fakten in Kapiteln in Kapiteln davor lassen sehr viel Raum für Interpretation. Deswegen in diesen Kapitel wird direkt wie auch in
Harvard-Konzept des Sachgerechten Verhandelns\cite{hk} erwähnt ist nur die direkten Alternativen und dagegen stehenden Drohungen betrachtet.\\
Die Verhandlungen werden durch alle in HKSV dargestellten Faktoren beeinflüst. Jedoch wenn das Gespräch in Phase "`Alternativen suchen"' angelangt
ist, dann ist es sehr gefährlich für beide Parteien, da schon vorher mindestens eine von mehreren Phasen gescheitert ist. Wir betrachten zunächst
welche von Faktoren überhaupt die Verhalndlung stören können. Folgende Liste stellt dar, welche Fehler passieren können:
\begin{itemize}
  \item Kommunikation ist gestört durch nicht einhlaten von \textbf{HASE}\footnote{\textbf{H}ören, \textbf{A}uthentizität, \textbf{S}prache,
  \textbf{E}mpathie}\cite{nk}-Prinzip
  \item Die vier Seiten einer Nachricht werden nicht eingehalten
  \item Störungen bei non-verbalen Kompetenz
  \item Die sieben Kommunikationsregeln\cite{nk}
  \item Einer von \textbf{ETHOS}\footnote{\textbf{E}conomical, \textbf{T}echnical, \textbf{H}uman, \textbf{O}rganisational, \textbf{S}ocial}\cite{nk}
  wird gestört
\end{itemize}
\section{Alternativen bezogen auf die Aufgabenstellung}
Es ist nicht einfach unter hier die Alternativen zu finden die beide Parteien akzeptiren können. Tabelle\ref{tab:interesseUndPositionen} in Kapitel
Analyse stellt es die Interessen und die Positionen von unterschiedlichen "`Personen"' in dieser Thema dar. Anhand dieser Tabelle wirden die
Alternativen Lösungen ausgearbeitet. \\

TODO 

\section{Überblick über die Alternativen Lösungen}

TODO